% SIAM Shared Information Template
% This is information that is shared between the main document and any
% supplement. If no supplement is required, then this information can
% be included directly in the main document.


% Packages and macros go here
\usepackage{lipsum}
\usepackage{amsfonts}
\usepackage{amsmath}
\usepackage{graphicx}
\usepackage{epstopdf}
\usepackage{algorithmic}
\usepackage{enumerate}
\usepackage{dsfont}

\ifpdf
  \DeclareGraphicsExtensions{.eps,.pdf,.png,.jpg}
\else
  \DeclareGraphicsExtensions{.eps}
\fi

% Add a serial/Oxford comma by default.
\newcommand{\creflastconjunction}{, and~}
\newcommand{\prob}[1]{\mathds{P} \left( #1 \right)}
\newcommand{\inv}[1]{#1^{-1}}
\newcommand{\F}{\mathcal{F}}
\newcommand{\norm}[1]{\left| #1 \right|}

\DeclareMathOperator*{\argmax}{arg\,max}
\DeclareMathOperator*{\argmin}{arg\,min}

% Used for creating new theorem and remark environments
\newsiamremark{remark}{Remark}
\newsiamremark{hypothesis}{Hypothesis}
\crefname{hypothesis}{Hypothesis}{Hypotheses}
\newsiamthm{claim}{Claim}

% Sets running headers as well as PDF title and authors
\headers{6.437 Final Project | Part I}{Juan M Ortiz}

% Title. If the supplement option is on, then "Supplementary Material"
% is automatically inserted before the title.
\title{6.437 Final Project | Write up I}

% Authors: full names plus addresses.
\author{Juan M Ortiz}

\usepackage{amsopn}
\DeclareMathOperator{\diag}{diag}
